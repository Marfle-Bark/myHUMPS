%
% This file is an example file demonstrating how to use the sigcomm-alternate class for papers
% submitted in partial fulfillment for assignments in the Computer Networking and Distributed Systems
% classes, CSE 4344 and CSE 5344/7344, at Southern Methodist University. The sigcomm-alternate 
% LaTeX class is used for submissions to CCR. It is a goal of at least CSE 5344/7344 that the students
% will submit at least paper submitted for an assignment to CCR for publication; therefore, this example
% file conforms to the CCR submission guidelines.
%
% Author: Cameron A. Keith 
% Creation Date: 8 October 2014
% Version: 0.3
%

\documentclass{sigcomm-alternate}
\usepackage{cite}
%---- The front matter contains packages being used, definitions, etc. including the title and author
%---- information for this work.

%---- Title
% The title of this work. For course assignments, the title must include the course number. 
\title
{
Analysis of a Stream Size Biased Medim Access Control Protocol
}

%---- Authors
% The authors are placed all within a single LaTeX author box. This is a CCR guideline that is not true of
% all ACM publications.
\numberofauthors{3}

\author{
% Each author is identified by name, simple affiliation, and email address. The command \alignauthor (no
% curly braces required) must precede each author name. The command \affaddr (curly braces required) 
% must precede each line in the affiliation. The command \email{ } (curly braces required) must by used to 
% identify the email address for each author. When more than three authors exist, a tabular environment will
% need to be used to align the authors across multiple lines.
\alignauthor Cameron A. Keith\\
\affaddr{Computer Science and Engineering Department\\
 	Southern Methodist University\\
	Dallas, Texas USA}\\
\email{ckeith@smu.edu}
%
\alignauthor Anna A. Carroll\\
\affaddr{Computer Science and Engineering Department\\
 	Southern Methodist University\\
	Dallas, Texas USA}\\
\email{aacarroll@smu.edu}
%
\alignauthor Dylan C. Fansler\\
\affaddr{Computer Science and Engineering Department\\
 	Southern Methodist University\\
	Dallas, Texas USA}\\
\email{dfansler@smu.edu}
%
\and
\alignauthor Ethan Busbee\\
\affaddr{Computer Science and Engineering Department\\
 	Southern Methodist University\\
	Dallas, Texas USA}\\
\email{ebusbee@smu.edu}
%\and  % use '\and' if you need 'another row' of author names
%------ Uncomment the following for editorial submissions to CCR
%\begin{tabular}{c}
%{\normalsize This article is an editorial note submitted to CCR. It has NOT been peer reviewed.}\\
%{\normalsize The authors take full responsibility for this article's
%technical content. Comments can be posted through CCR Online.}
%\end{tabular}
%}
}

%--- Begin the Document
\begin{document}

%---- Create the title
\maketitle

%---- Abstract
% The abstract is a 100 word summary of the problem addressed in the paper, the basic thesis and/or 
% the contributions of the paper, and the primary results and conclusions presented in the paper.
\begin{abstract}
%Current Medium Access Control, MAC, Protocols are based on the premise that every device 
%has an equal probability to transmit to their data across a network. This method 
%attempts to reduce the collisions in the network by having all users who want to send 
%a packet wait a random time interval between 0 and a minimum starting back-off,
% and increasing a particular senders minimun back-off each time their packet does not
% sent. This paper attempts to discover if a Wireless MAC protocol based off of using the 
%current frame number as the upper bound of the random number function in an attempt to 
%see if a biased priority can improve the average complete time of stream transmittion time accross a 
%particuar network.
This paper looks to improve the overall network efficiency by using a MAC protocol that biases towards Streams of shorter sizes by basing the initial back off of the packet transmission off the current frame number being sent instead of a system declared minimum initial value. This approach should improve the average time to transmit data from all nodes on the network.
\end{abstract}

%----Main Body of the Paper

\section{Introduction}
{
Connecting to the internet has never been easier than it is in today's world. The ability to join a Wireless network is as simple as a few clicks of a mouse or taps on a phone, and suddenly the world is at your finger tips. The ability for people to connect from around the world is increasing at an ever expanding rate with wireless access being introduced to the most remote places. With this sudden surge in increased accessability and user interaction comes the question of is the current connection the best we have or can we improve? \\

Current MAC protocols operate with the assumption that all data being sent should be equal regardless of the type of data or the sender \cite{5340799, FairScheduling}.
}

\section{Background}
{
So far, very little work has been done on the optimization of unfair Medium Access Protocols; most work has been done on solving the unfair MAC problem while reducing collisions. For wireless networks, a principle problem is the “hidden terminal problem”. In a wired network, it is possible to sense when another node is being sent, and a collision will occur \cite{869217}. In a wireless network, however, this cannot be done, resulting in the need for an alternative method of handling collisions \cite{6574961}. Thus, wireless MAC protocols like Slotted-Aloha were created.  In the Slotted Aloha Protocol, a node can only be sent at the beginning of a time slot \cite{5340799}. This ensures that one node can finish sending before another one is sent, reducing the number of collisions.  Under a network with a light load, this approach has a low chance of collision. However, collisions can still occur if two nodes are sent in the same timeslot. Under heavy loads, the probability that a node will be sent in the same timeslot as another node will increase. Most research into this problem is based on reducing collisions while keeping the distribution of sent nodes fair.  An improvement on the Slotted Aloha Protocol is the Frameless ALOHA protocol, which uses a random access scheme to decide which nodes should be sent in which spot \cite{6336861}. Another attempt at improving the Slotted-ALOHA protocol is the Generalized Slotted ALOHA protocol, in which nodes are sent according to their probability of being transmitted successfully. However, an issue with this protocol is that if every node is attempting to maximize its own transmission rate, then the network can jam. \cite{4548143 }.  The Multiple Access with Collision Avoidance for Wireless (MACAW) protocol is one attempt at improving on the Slotted-Aloha Protocol and solving the hidden terminal and unfair MAC problems. MACAW introduces “per stream” fairness, in which every stream sent on the network is treated equally. This contrasts our proposal in that we will attempt to create an optimal MAC protocol by prioritizing certain streams, thus creating an unfair MAC protocol that optimizes the network.\\

A similar task to creating an unfair MAC protocol that everyone follows is how can a network detect when a single user, or a small group of users is being unfair and the process of handling them. With this protocol a sender transmits an RTS (Request to Send) after waiting for a randomly selected number of slots in the range [0; CW].After the initial transmittion between hosts, the recieving host sends with their acknowledgement a random value that the sender then uses as the back off counter for each subsiquent transmittion during the stream. \cite{Kyasanur} With this protocol, if a recieving node recieves a packet before the approiate number of frames has passed then the sending host is not obeying the Protocol and can then be handled accordingly.
}

\bibliographystyle{plain}
\bibliography{unfairMACPaperbib}  % unfairMACPaper_bib.bib is the name of the Bibliography in this case

\balancecolumns
%---- End the Document
\end{document}




