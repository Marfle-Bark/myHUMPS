%
% This file is an example file demonstrating how to use the sigcomm-alternate class for papers
% submitted in partial fulfillment for assignments in the Computer Networking and Distributed Systems
% classes, CSE 4344 and CSE 5344/7344, at Southern Methodist University. The sigcomm-alternate 
% LaTeX class is used for submissions to CCR. It is a goal of at least CSE 5344/7344 that the students
% will submit at least paper submitted for an assignment to CCR for publication; therefore, this example
% file conforms to the CCR submission guidelines.
%
% Author: Cameron A. Keith 
% Creation Date: 8 October 2014
% Version: 0.3
%

\documentclass{sigcomm-alternate}
\usepackage{cite}
%---- The front matter contains packages being used, definitions, etc. including the title and author
%---- information for this work.

%---- Title
% The title of this work. For course assignments, the title must include the course number. 
\title
{
Analysis of a Stream Size Biased Medim Access Control Protocol
}

%---- Authors
% The authors are placed all within a single LaTeX author box. This is a CCR guideline that is not true of
% all ACM publications.
\numberofauthors{3}

\author{
% Each author is identified by name, simple affiliation, and email address. The command \alignauthor (no
% curly braces required) must precede each author name. The command \affaddr (curly braces required) 
% must precede each line in the affiliation. The command \email{ } (curly braces required) must by used to 
% identify the email address for each author. When more than three authors exist, a tabular environment will
% need to be used to align the authors across multiple lines.
\alignauthor Cameron A. Keith\\
\affaddr{Computer Science and Engineering Department\\
 	Southern Methodist University\\
	Dallas, Texas USA}\\
\email{ckeith@smu.edu}
%
\alignauthor Anna A. Carroll\\
\affaddr{Computer Science and Engineering Department\\
 	Southern Methodist University\\
	Dallas, Texas USA}\\
\email{aacarroll@smu.edu}
%
\alignauthor Dylan C. Fansler\\
\affaddr{Computer Science and Engineering Department\\
 	Southern Methodist University\\
	Dallas, Texas USA}\\
\email{dfansler@smu.edu}
%
\and
\alignauthor Ethan Busbee\\
\affaddr{Computer Science and Engineering Department\\
 	Southern Methodist University\\
	Dallas, Texas USA}\\
\email{ebusbee@smu.edu}
%\and  % use '\and' if you need 'another row' of author names
%------ Uncomment the following for editorial submissions to CCR
%\begin{tabular}{c}
%{\normalsize This article is an editorial note submitted to CCR. It has NOT been peer reviewed.}\\
%{\normalsize The authors take full responsibility for this article's
%technical content. Comments can be posted through CCR Online.}
%\end{tabular}
%}
}

%--- Begin the Document
\begin{document}

%---- Create the title
\maketitle

%---- Abstract
% The abstract is a 100 word summary of the problem addressed in the paper, the basic thesis and/or 
% the contributions of the paper, and the primary results and conclusions presented in the paper.
\begin{abstract}
%Current Medium Access Control, MAC, Protocols are based on the premise that every device 
%has an equal probability to transmit to their data across a network. This method 
%attempts to reduce the collisions in the network by having all users who want to send 
%a packet wait a random time interval between 0 and a minimum starting back-off,
% and increasing a particular senders minimun back-off each time their packet does not
% sent. This paper attempts to discover if a Wireless MAC protocol based off of using the 
%current frame number as the upper bound of the random number function in an attempt to 
%see if a biased priority can improve the average complete time of stream transmittion time accross a 
%particuar network.
This paper looks to improve theoverall network effiency by using a MAC protocol that biases towards Streams of shorter sizes by basing the initial back off of the packet transmittion off the current frame number being sent instead of a system declared minimum initial value. This approach should improve the average time to transmit data from all nodes on the network.
\end{abstract}

%----Main Body of the Paper

\section{Introduction}
{
%Current MAC protocols for 802.11 use a variation of the ALOHA Protocol known as p-persistant CSMA/CA, or Carrier Sense Multiple Access with Collision Avoidance. In this protocol, once a node has assembled a message, it checks to see if the channel is idle. If not, the node waits a random backoff time before checking the channel again. Once the node is able to send through the channel, it transmits a RTS (request to send) frame to the destination node, and waits for a CTS (clear to send) frame from the destination node. If the CTS frame is not received, then the node waits a random backoff time and attempts to send again. If the CTS frame is received, the node transmits its message to its destination. With this approach, all frames are given an equal chance to transmit. However, in this protocol, the packet delay tends to increase with the amount of stations (or nodes) on the network. Instead of using a fair approach with respect to each device on the network, we plan to explore the possibility of using a biased treatment of frames in the networks to increase the overall efficiency of the network.\\
%
%Most work on biased MAC protocols is focused on either acting unfairly in order to increase one node's network performance at the cost of the performance of the other nodes on the network, or is done on keeping a node acting unfairly from destroying the network and discouraging that node from acting unfairly any more. It is our hypothesis that a biased protocol can be created that will actually increase the performance of the network. We plan to do this by examining different ways of prioritizing network traffic according to packet size, type, or even the size of the file to be transmitted. 
}

\section{Background}
{

}

\bibliographystyle{plain}
\bibliography{unfairMACPape_bib}  % unfairMACPaper_bib.bib is the name of the Bibliography in this case

\balancecolumns
%---- End the Document
\end{document}




